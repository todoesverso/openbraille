% capitulo que contiene el desarrollo tecnico del disposotivo (hardware y
% software), convencionalmente fue decidido iniciar por el dise\~no de hardware
% y luego continuar con la programacion del hardware (software embebido p
% firmware) y finalmente el programa que se ejecuta en la PC (software)

\chapter{Propuesta T\'ecnica}
% Comentario de este capitulo 

\section{Introduccion} 

% Breve introduccion para mencionar como se van a desarollar los temas 
% incluidos en este capitulo, que se propone obtener y basicamente presentar
% los tres bloques mas importantes: harware, firmware y software.
% la organizacion no es definitiva y puede cambiar su orden, se propone un
% draft inicial.
% TBD: Verificar la redaccion y agregarle formalismo tecnico, reorganizar
% revisi\'on por parrafos pendientes para volver a redactar las ideas escritas.

En este capitulo se explica a continuacion el procedimiento t\'ecnico
desarrollado para construir una impactadora de codigo braille.
Como primera propuesta, se comienza en investigar y adaptar una impresora a
matriz de puntos a modo de prototipo para lograr imprimir codigo sobre papel
comun, sin alterar demasiado su funcionamiento regular y mantieniendo
preferentemente los elementos actuadores.
Se realizan algunas adaptaciones mecanicas y electricas, y mediante un sencillo
programa se traduce un texto de sample enviando el conjunto de caracteres
codificados para actuar sobre las funciones de la impresora.
Ademas se presenta el dise~no de la electr\'onica de control para el
accionamiento de los elementos meacnicos, utilizando el microcontrolador
PIC18F4550. Este procesador incorpora un modulo USB y el funcionamiento del SIE
ser]'a explicado en detalle.
Tambi\'en se incluir\'an las simulaciones y mediciones necesarias para la
calibraci\'on y puesta a punto de los dispositivos mecanicos (velocidad de los
motores electricos, niveles de corriente y carga, tiempo de trabajo necesario
para el percutor imprima los puntos y regrese a su posici\'on de origen, etc).
% ...
El percutor consiste en un sistema mecanico sencillo de un unico elemento
impactador que ser\'a accionado mediante una corriente electrica. 
No se profundizar\'an los detalles del dise\~n o mecanico ya que sale fuera
del alcance de este trabajo, y da su prioridad sera el ultimo elemento a
construir.
Un circuito de control adicional maneja el percutor en la forma que adecuada,
en la programaci\'on del microcontrolador se realizan los cambios necesarios 
para modificar el pulso (o pulsos) que se necesiten para enviar la se\~nal de 
activacion a los diferentes tipos de solenoides.
Los motores ser\'an abordados de forma independiente, para controlar el
movimiento 
rotacional y transversal, de modo de ubicar el carro en el lugar de impacto. 
Una calibracion adicional sera necesaria para ajustar la velocidad
% que sera resuelta con un circuito de prueba especialmente contruido para este
% fin (obtener los parametros de funcionamiento mecanico)
% ...
% ????

\section {Dise\~no electr\'onico (Hardware)}

% modificar el titulo anterior en caso de no ser apropiado, puede encontrarse
% una referencia bibliografica para respaldar la eleccion (conveniente pero no
% excluyente)
% ------------------------------------------------------------------------
%
% Contenido principal de la seccion
%
% ------------------------------------------------------------------------
% \subsection{Firmware}
% Respaldar el parentesis anterior con la referencia de la investigacion
% realizada 
%
\section {Dise\~no l\'ogico (Software)}

% nuevamente modificar el titulo anterior si no es apropiado, puede encontrarse
% una referencia bibliografica para respaldar la eleccion (conveniente pero no
% excluyente)
% ------------------------------------------------------------------------
%
% Contenido principal de la seccion
%
% ------------------------------------------------------------------------
%
%\section {Nombre_seccion}
%
% ------------------------------------------------------------------------
%
% Contenido principal de la seccion
%
% ------------------------------------------------------------------------
% 
