\chapter{Econom\'ia}
%
%
Debido a la caragter\'istica meramente investigativa de \'este trabajo, solo
se exponen en este cap\'itulo posibles modelos de negocios alrederor del
\emph{Software Libre}, y algunos nichos de mercado que estan siendo explotados
en la actualidad.\\

\section{Modelos de negocios}
La organizaci\'on FLOSSMetrics 
(Free / Libre and Open Source Software Metrics
\footnote{V\'ease - \url{http://www.flossmetrics.org/}})
, que forma parte de la Flossquality (Open source quality research
\footnote{V\'ease - \url{http://www.flossquality.eu/}})
, ambos financiados por la Comisi\'on Europea
\footnote{Vease -
\url{http://cordis.europa.eu/fetch?CALLER=PROJ_IST&ACTION=D&RCN=79449}}
, postula nueve categorizaciones
\footnote{V\'ease -
\url{http://smeguide.conecta.it/index.php/6._FLOSS-based_business_models}} 
de modelos de negocios potenciales.


%%%%%%%%%%%%%%%%%%%%%%%%%%%%%%%%%%%%%%%%%%%%%%%%%%%%%%%%%%%%%%%%%%%%%%%%%%%%%%%
\subsection{Financiaci\'on externa de empresas}\footnote{En ingl\'e
\emph{Externally funded ventures}}
%
Esta categoria contempla los grupos o compa\~nias que desarrollan
\emph{Software Libre} siendo financiados por una organizaci\'on externa.
Normalmente es \'esta organizacion externa quien pone las gu\'ias
requerimientos para el proyecto. \'Esta categoria es luego desglosada en tres
partes segun el origen de la financiaci\'on.

\subsubsection{Financiaci\'on P\'ublica}\footnote{En ingl\'es \emph{Public
funding}}
%
Se refiere a aquellos proyectos que son financiados por entidades p\'ublicas.
\'Estos suelen ser proyectos cint\'ificos o desarrollo de estandares cuyo fin
\'ultimo no necesariamente es generar ganancias econ\'omicas si no m\'as bien
algun beneficio social.

\subsubsection{Financiaci\'on ``Necesidad de mejora''}\footnote{En ingl\'es
\emph{``Needed improvement'' funding}}
%
\'Esta es la situaci\'on donde una compa\~nia le paga a un grupo u otro
compa\~nia para que mejore o adapte (y quiz\'a m\'as tarde de soporte) de un
proyecto libre existente.

\subsubsection{Fundaci\'on Indirecta}\footnote{En ingl\'es \emph{Indirect
funding}}
%
Muchas veces ciertas compa\~nias financias proyectos de \emph{Software Libre}
como estarategia de negocio, para obtener, de manera indirecta, algun r\'edito
econ\'omico. El ejemplo m\'as comun es cuando un fabricante de \emph{hardware}
financia a alguien m\'as (muchas veces son sus mismos empleados trabajando
para proyectos de \emph{Software Libre}) para que escriba los \emph{drivers}
para un sistema operat\'ivo libre (como lo puede ser GNU/Linux). 

\subsection{Uso Interno}\footnote{En ingl\'es \emph{Internal use}}
%





