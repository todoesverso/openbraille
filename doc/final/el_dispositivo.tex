% El objetivo de este capitulo es presentar a grandes rasgos el dispositivo 
% y luego sus detalles. Sus caracteristicas  mas importantes, explicar el 
% funcionamiento, alternativas, diagramas, comparaciones, ventajas y desventajas
% de cada opcion, como presentar tambien los detalles pertinentes, los factores
% principales y prioritarios.
\chapter{El dispositivo}
% Comentario de este capitulo 
\section{Introduccion} 
% Breve introduccion para presentar el dispositivo visto como prototipo que
% tiene que cumplir determinada funcion, planteando los requerimientos y
% ampliando en detalles gradualmente a medida que se avanza en la lectura
% el enfoque tiende a ser teorico antes de abordar cualqueir conclusi�n parcial
% pr�ctica
La impresora Braille o embosser (estampadora) es un dispositivo que realiza un
estampado en relieve de arreglo de puntos sobre un papel ductil, de manera 
autom�tica (es decir no-manual).
% incluir un diagrama en bloques simplificado del sistema completo aqui mismo.
\section {El sistema Braille}
% Draft inicial de esta seccion. Ampliarla de acuerdo a la relevancia e incluir
% figuras y gr�ficos
El sistema Braille es un m�todo utilizado por personas ciegas para leer y 
escribir. Fue ideado en 1821 por el franc�s Louis Braille
Se basa en un m�todo de comunicaci�n desarrollado y perfeccionado por Charles 
Barbier en respuesta a la demanda de Napole�n de un c�digo que los soldados 
pudieran usar para comunicarse en silencio y sin luz en la noche y se lo llam� 
Night writing. El sistema de Barbier era demasiado complejo para los soldados de 
aprender, y fue rechazada por los militares. En 1821 visit� el Instituto Nacional 
para Ciegos, en Par�s, Francia, donde conoci� a Louis Braille. Braille identificado 
el mayor defecto de c�digo, que es que el dedo de la mano humana no puede abarcar 
todo el s�mbolo sin moverse, y as� no puede pasar r�pidamente de un s�mbolo a otro.
Su modificaci�n fue utilizar una celda de 6 puntos - el sistema Braille - que 
revolucion� la comunicaci�n escrita de los ciegos.
Cada c�lula (o celda) braille o car�cter se compone de seis posiciones de puntos,
dispuestos en un rect�ngulo que contiene dos columnas de tres puntos cada uno. Un
punto puede ser colocado en alguna de las seis posiciones para formar sesenta y 
cuatro (2^6) permutaciones, incluido el arreglo de puntos que no se coloca. Una 
permutaci�n puede ser descrita nombrando las posiciones en que se disponen los 
puntos: Las posiciones est�n universalmente numerados de 1 a 3, de arriba a abajo, 
a la izquierda, y 4 a 6, de arriba a abajo, a la derecha. 
%Por ejemplo, puntos 1-3-4 describir�a una celda con tres puntos planteados, en la parte superior e inferior de la columna de la izquierda y en la parte superior de la columna de la derecha, es decir, la letra m. 
Las l�neas horizontales de texto en Braille est�n separados por un espacio a
fin de que los puntos de una l�nea puede ser diferenciada de la de texto en 
braille por encima y por debajo. La puntuacion est� representada por su propio 
conjunto de caracteres �nico.
% incluir figuras para facilitar la explicacion

%\section {Nueva_seccion}
% nuevamente modificar el titulo anterior si no es apropiado, puede encontrarse
% una referencia bibliografica para respaldar la eleccion (conveniente pero no
% excluyente)
%
% 
% 
