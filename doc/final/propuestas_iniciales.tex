% Uno de los capitulos iniciales del cuerpo del trabajo, contiene una
% introduccion del capitulo y el marco teorico de los diferentes enfoques
% posibles como formas de implementar el trabajo.
\chapter{Propuestas iniciales}
% Comentario de este capitulo 
\section{Introduccion} 
% Explicar que son los puntos que se mencionan a continuacion 
\section {Enfoque estilo industrial}
% modificar el titulo anterior en caso de no ser apropiado, puede encontrarse
% una referencia bibliografica para respaldar la eleccion (conveniente pero no
% excluyente)
Este enfoque de proyecto se basa en generar un producto para ser inyectado en
el mercado de las impresoras braille, mediante el diseño de todo el esquema de
produccion, la determinacion de los recursos que intervienen como factores de
la produccion y del ciclo de vida del producto. Corresponde a un esquema clasico
y convencional que consiste en la explotacion de materia prima, recursos
humanos, infraesctructura y soporte financiero para establecer un producto
industrial. El factor economico interviene de forma imperativa en este
paradigma de fabricacion y la motivacion es enteramente lucrativa. 
Generalmente estos aspectos son los que definen la economia de escala, que
junto con los factores de la produccion definen la rentabilidad.
El diseño se inicia con un estudio de la demanda y la inclusion del producto en
el mercado existente de impresoras (incluyendo los circuitos de distribucion, 
comercializacion y servicios), hasta llegar al consumidor final.
El beneficio se define en funcion del precio (valor agregado del producto
final) y de los costos que intervienen en la cadena de produccion. 
El fin de este modelo es la comercializacion, sustentado unicamente por la
competencia en el mercado. 
Esta caracteristica conduce a la complejidad no necesaria del dispositivo para
hacerla competente dentro del mercado. Por ejemplo, incluir una caracteristica
de compatibilidad con un producto cualquiera ("Bista compatible" por ejemplo) 
que tiene unico fin de aumentar las ventas, incidiendo en los consumidores al
manipular sus desiciones de consumo (competencia agresiva).
A su vez quedan implicitos altos costos por solo uso de marca y pago de 
licencias, y sin impacto en la calidad como producto.
Como consecuencia de adoptar este modelo de trabajo, los factores que conducen
a elevado precio final y la falta (o lentitud) de subsidios nacionales
para financiar el precio pueden llegar a injustificar completamente la 
realizacion de este proyecto por no ser rentable (y de hecho asi sucede en
Argentina, no existe una sola fabrica de impresoras Braille).
% Respaldar el parentesis anterior con con la referencia de la investigacion
% realizada 
Ademas no soluciona el problema estructural ya que el alcance de estas
impresoras seria para sectores sociales de alto poder adquisitivo.
\section {Enfoque estilo investigacion financiada}
% nuevamente modificar el titulo anterior si no es apropiado, puede encontrarse
% una referencia bibliografica para respaldar la eleccion (conveniente pero no
% excluyente)
Este enfoque se centraliza en adoptar una metologia de trabajo bajo el concepto
de investigacion clasica, avalada por una institucion, estado  o empresa como
inversion en materia economica. Los costos son mas flexibles, pero pueden ser
mas limitados segun la escala (al tratarse de trabajos por producto o por
prototipo, la financiacion se establece en base a los resultados que influiran
en la inversion y no tanto en la necesidad inmediata).
Si bien hereda caracteristicas del modelo industrial (por ejemplo la
competencia aunque en este caso menos agresiva), la sustentbilidad del modelo
esta establecida por el valor que la inversion resulta y se traduce en
beneficio economico a largo plazo (generalmente conocida como I+D)
lo que en definitiva define en este caso el valor agregado. Los costos
nuevamnete aplican a recursos, patentes intelectuales, licenciamiento
propietario y uso de marcas.
El modelo apunta a desarrollar un prototipo que resuelva una necesidad 
haciendo uso de los recursos establecidos por el organismo que financia dicho
producto. Tampoco existe libertad en la eleccion de las herramientas mas
convenientes o una conveniencia tecnica determinada (generalmente se adoptan
elementos de uso masivo y nunca se pregunta por que) y como resultado 
el proyecto terminado hereda estos atributos restrictivos que terminan
impactando en la libre eleccion de recursos y tecnicas optimas de
implementacion. 
\section {Enfoque alternativo}
% nuevamente modificar el titulo anterior si no es apropiado, puede encontrarse
% una referencia bibliografica para respaldar la eleccion (conveniente pero no
% excluyente)
%
% Este parrafo es el fundamental y debe ser redactado cuidadosamente señalando
% las caracteristicas esenciales de un desarrollo autosustentable, el valor
% agregado es el beneficio social y principalmente este enfoque no trabaja
% sobre un mecanismo de competencia sino de cooperatividad, que le da la
% sustentabilidad suficiente para evolucionar.
% observar fundamentando este punto la caracteristica comunitaria del SL y el
% fenomeno de formacion de comunidades naturalmente
% tambbien señalar que implica un riesgo por ser un enfoque diferente y
% original (lo no convencional despierta miedo) y el riesgo es el rechazo por 
% 
\section{Enfoque utilizado} 
El trabajo fue realizado siguiendo las ideas establecidas en el ultimo enfoque
descrito, que se justifica por las siguientes razones:
% enumerar y dar razones de por que seguir el camino no convencional
