\documentclass[12pt,a4paper]{report}
%Encabezado segun reglas de trabajo final

% Paquetes usados
\usepackage[latin1]{inputenc}
\usepackage[spanish]{babel}
\usepackage{fancyhdr}
%\usepackage{fullpage}
\usepackage{graphicx}
\usepackage{psfig}
\usepackage{subfigure}
\usepackage{color}
\usepackage{fancybox}
\usepackage{graphics}
\usepackage{amssymb}
\usepackage{hyperref}
\usepackage[centerlast,small,bf]{caption2}
\usepackage{template}

% Formateador de  portada
\newcommand{\THESISNAMEA}{Impresora Braille - Implementaci\'on libre }
\newcommand{\THESISAUTHORA}{\hspace*{1in}Rosales Victor Hugo\hspace*{1.2in}787774}
\newcommand{\THESISAUTHORB}{\hspace*{1in}German Horacio Sanguinetti\hspace*{1.2in}977969}
\newcommand{\THESISDATE}{Diciembre de 2008}
\newcommand{\THESISYEAR}{2009}
\newcommand{\THESISSINODALA}{Ing. John Coppens}
%-----------------------------------------------------------------------------------------

\title{\THESISNAMEA}
\date{\THESISDATE}

% No se que hace esto, por las dudas lo saque.
%\setcaptionwidth{13cm}

\psfigurepath{./img/}

% ----------------------------------------------------------------------------
\begin{document}

%------------------------------------------------------------------------------------------
% Portada principal

\sloppy
%\newpage
\thispagestyle{empty}
%\vspace*{-0.5in}
\begin{center}
%\hspace*{-0.6in}
{\bf \mbox{Universidad Cat\'olica de C\'ordoba}}\\
%{\bf CAMPUS ???? } \\
%\vspace*{0.15cm}
%\hspace*{-0.1in}

%\vspace*{0.7in}
{\bf TRABAJO FINAL} \\
%\vspace*{0.7in}
%{\bf VISION COMPUTACIONAL}\\
%\vspace*{0.4in}
{\bf \THESISSINODALA} \\
%\vspace*{1in}

\begin{large}{\bf \THESISNAMEA}\end{large} \\
\end{center}
%\vspace*{0.4in}
%\vspace*{0.4in}
{\bf \THESISAUTHORA} \\
{\bf \THESISAUTHORB} \\
%\vspace*{0.5in}

%\begin{center}
%\begin{figure}[h]
%\centerline{\psfig{file=example.eps,width=3in}}
%\end{figure}
%\end{center}

\begin{center}
%\vspace*{0.5in}
{\bf C\'ordoba, Argentina., \THESISDATE}
\end{center}


\sloppy
\newpage
\thispagestyle{empty}

\begin{center}
\section*{Agredecimientos}
... a todos aquellos que hicieron, hacen y har\'an que mi vida valga la pena
...
\end{center}

% -----------------------------------------
\tableofcontents  %% Genera indice general.
% -----------------------------------------

\chapter{Pr\'ologo} %%%%%%%%%%%%%%%%%%%%%%%%%%%%%%%%%%%%%%%%%%%%%%%%%%%%%%%%%%%
%%%%%%%%%%%%%%%%%%%%%%%%%%%%%%%%%%%%%%%%%%%%%%%%%%%%%%%%%%%%%%%%%%%%%%%%%%%%%%%
% Prólogo, donde se definan sus alcances, propósitos y objetivos.
% Requerido por reglamento de trabajo final 

Durante el transcurso del trabajo se describe el desarrollo e implementaci\'on
de la parte electr\'onica de una impresora braille. El objetivo final del
mismo es proveer una posible soluci\'on de bajo costo, tanto de producc\'ion
como de armado. Para ello, se estudiar\'an y analizar\'an todas aquellas
alternativas disponibles que puedan satisfacer los requerimientos
planteados.\\

La principal premisa con que se aborda el desarrollo de este trabajo es que
mediante soluciones b\'asicas y simples, es posible concretar dise\~nos
robustos y funcionales, siempre y cuando se hayan elegido las herramientas y
condiciones correctas. Para lograr esto, gran parte del trabajo se centra en
investigar alternativas de desarrollo no convencionales en la industria
electr\'onica.

% Prólogo, donde se definan sus alcances, propósitos y objetivos.



% Segun reglamento de trabajo final
% Desarrollo, en esta parte se deben incorporar todos los conocimientos Ingenieriles
% que permitan superar los problemas planteados.
% No necesariamente se deben llegar a resultados positivos, puede suceder que se
% obtengan resultados negativos y son tan válidos como los anteriores, no olvidar
% que la idea del T. F. es como el primer trabajo profesional.
% En este punto es importante hacer intervenir la mayor cantidad de materias que se
% han cursado, porque fundamentalmente estamos hablando de un trabajo
% integrador.

%---------------------------------------------------------------------------
% Cada capitulo debe ser un archivo separado para mejor mantenimiento.
% Ademas cada capitulo debe tener
% \chapter{title}
%	- Prologo del capitulo sin ningun formato especial(?)
%	- Secciones....
%---------------------------------------------------------------------------


%\section{Software Libre}

El Software Libre, es aquel que le asegura cuatro libertades b\'asicas al
usuario. La libertad de usa el programa, de estudiar su funcionamiento, de
compartirlo y la libertad de mejorarlo y distribuirlo.

\subsection{Historia}
El concepto de Software Libre (Free Software) comienza a gestarse en la 
d\'ecada del 60/70. En ese entonces el software no era considerado un
producto, sino m\'as bien un a\~nadido que formaba parte del hardware. Por esta
raz\'on las personas que trabajaban con software compart\'ian sus c\'odigos
fuentes libremente. A finales de los 70, las compa\~n\'ias de software
comenzaron a implementar licencias a sus programas con ciertas restricciones. 
Surg\'ia entonces la necesidad de tener un sistema operativo libre donde
correr aplicaciones libres.


\subsection{GNU}
En 1984, un estudiante del Instituto de Tecnolog\'ia de Massachusetts
(Massachusetts Institute of Technology - M.I.T), llamado Richard Stallman, 
comenz\'o a desarrollar el proyecto G.N.U (GNU's not UNIX).
\'Este, pretendia ser un reemplazo libre de los sistemas propietarios que 
estaban surgiendo en aquella \'epoca.
Sus motivos\footnote{V\'ease ''Manifiesto GNU'' - 
http://www.gnu.org/gnu/manifesto.es.html} eran diversos, pero principalmente 
cre\'ia que era necesario desarrollar un entorno l\'ibre para las personas.
Para llevar acabo su proyecto se bas\'o en UNIX, un sistema operativo 
portable multiusuario y multitarea. Junto con la Colecci\'on de Compiladores
GNU (GCC)\footnote{Previamente llamado -GNU C Compiler- puesto que solo
compilaba codigo C. Versiones m\'as recientes soporta varios leguajes. V\'ease
- http://gcc.gnu.org}, un editor de texto y todo un stack de software,
comenz\'o a idear el proyecto GNU.

\subsection{GNU/Linux}
El proyecto GNU tomaba forma pero le faltaba un componente muy importante; el
kernel. 
En 1991, un estudiante finland\'es llamado Linus Torvalds, liber\'o un kernel
basado en UNIX, que luego pasar\'ia a formar parte de lo que hoy se conoce como
GNU/Linux.
GNU/Linux es un sistema operativo completo con un stack de software que
satisface la mayoria de las necesidades de un usuario m\'as un potente kernel
mutiplataforma.\
Lo que hace a GNU/Linux (y a la mayoria de los programas libres) versatiles,
robustos y estables, es que se basan en estandares abiertos y libres, como lo
son \emph{SUS}\footnote{Single Unix Specification - V\'ease -
\url{http://en.wikipedia.org/wiki/Single_UNIX_Specification}},
\emph{POSIX}\footnote{V\'ease - \url{http://en.wikipedia.org/wiki/Posix}} o
\emph{IEEE}\footnote{V\'ease - \url{http://www.ieee.org}}.

\subsection{Concepto de Software Libre}
Luego del proyecto GNU, Richard Stallman fund\'o la Fundaci\'on para el 
Software Libre (Free Software Foundation)
\footnote{V\'ease - http://www.fsf.org} quien se encarga (entre otras cosas) de
mantener la definici\'on del concepto\footnote{V\'ease -
http://www.fsf.org/licensing/essays/free-sw.html} de Software Libre.

\begin{quote}
``Free software'' is a matter of liberty, not price. 
To understand the concept, you should think of ``free'' as in ``free speech,
'' not as in ``free beer.''
\end{quote}

% Poner en cursiva o algo para resaltar que es una traduccion
El Software Libre es una cuesti\'on de libertad no de precio\footnote{\'Esta
aclaraci\'on surge debido a que en ingl\'es, \emph{free} significa tanto
\emph{libre} como \emph{gratis}.}. Para entender el concepto debe pensar libre
(\emph{free}) como en libre discurso no como cerveza gratis.\\

Para que un programa sea Software Libre, debe garantizarle cuatro libertades
b\'asicas al usuario:

\begin{itemize}
\item[Libertad 0:] Libertad de ejecutar el programa con cualuier prop\'osito
\item[Libertad 1:] Libertad de estudiar como funciona el programa y de
adaptarlo
a tus necesidades. \emph{Acceso al codigo fuente es un pre-condici\'on para
\'esto}
\item[Libertad 2:] Libertad de redistribuir copias del mismo para poder ayudar
a tu vecino.
\item[Libertad 3:] Libertad de mejorar el programa y pulicar los cambios para
que toda la comunidad se beneficie de ellos. \emph{Acceso al codigo fuente es 
un pre-condici\'on para \'esto}
\end{itemize}


\subsection{El Software Libre en la ingenieria} 
% No me gusta como suena el titulo este
Existen en la actualidad tecnologias o herramientas libres para lidiar con la
mayoria de los problemas de la ingenieria. Pero que, debido a su propia
naturaleza libre u \emph{open source}, carecen de publicidad suficiente como
para competir con sus alternativas \emph{propietarias}. \\

No obstante su condicion libre, la mayoria de estas tecnologias estan a la
altura de sus pares \emph{propietarios}. Tanto es asi que existen empresas que
se dedican exclusivamente a este tipo de tecnologias\footnote{Vease -
http://www.redhat.com/ - http://code.google.com/opensource/}, y otras grandes
empresas como Intel\footnote{Vease - http://software.intel.com/sites/oss/},
IBM\footnote{Vease - http://www.ibm.com/developerworks/opensource/} y
Motorola\footnote{Vease - https://opensource.motorola.com/}, trabajan
activamente en proyectos libres.\\

% Requerido por reglamento de trabajo final 
\chapter{Diagn\'ostico}

% Aca va una descripcion y analisis del problema abordado, un breve contenido
% de fundamento social y planteo de las necesidades.
% Incluir las razones qeu motivan este trabajo y se puede incluir una breve
% aproximacion historica.

% es recomendable dejarlo casi al final del trabajo

% Diagnóstico, en el mismo se mostrará la problemática actual, las
% dificultades a superar y todo aquello que se considere de interés y
% necesario para comprender el propósito que nos impulsa a tratar el tema.

En la actualidad existen muchas empresas que fabrican dispositivos de
impresi\'on braille, pero debido a el tama\~no reducido del nicho de mercado
en el que se encuentran, sus precios suelen ser muy elevados y no est\'an al
alcance del ciudadano medio.\ Y al pertenecer, de una forma u otra, a un sector
tecnol\'ogico, existe una gran competencia en cuanto al avance de sus
tecnolog\'ias. Esto conlleva a que fabriquen dispositivos con muchas
funcionalidades, prestaciones y de gran performance, dejando de lado dise\~nos
sencillos y meramente funcionales que har\'ian al producto menos costoso.\\

Otro problema que presentan estos dispositivos, es que, a falta de est\'andares
de impresi\'on braille, cada fabricante provee su propia soluci\'on de
software que suele ser un costo extra en algunas ocasiones.\ 
Por este mismo motivo el soporte que proveen suele limitarse a un \'unico
sistema operativo\footnote{Normalmente Microsoft Windows.} forzando al usuario
a comprar una licencia del mismo e incluso en muchos casos un \'unico
procesador de texto\footnote{Normalmente Word de la suite Microsoft Office.}.\\

Se encuentra tambi\'en dicha industria embebida en modelos de desarrollo
privativo, haciendo imposible al usuario final a agregar sus propios cambios
bas\'andose en sus necesidades particulares.\ Si bien el modelo de desarrollo
privativo es uno de los mas usados en todas las industrias, existen varias que
se encuentran, ya sea en etapas de exploraci\'on o producci\'on, trabajando con
modelos \emph{C\'odigo Abierto}\footnote{Del ingl\'es \emph{open-source}} o
incluso \emph{Software Libre}\footnote{Del ingl\'es \emph{ Free Software}},
siendo esto un lujo que la industria de dispositivos de impresi\'on braille no
puede darse debido mayormente a su tama\~no.\\

Las problem\'aticas antes planteadas hacen que un posible mercado nacional de
estas tecnolog\'ias sea pr\'acticamente imposible, por lo que las impresoras
braille deben ser adquiridas en el exterior o mediante un importador.\\

% Todo esto ###### <---buscar una palabra mejor
Todo esto termina en que los usuarios finales deben gastar una importante suma
de dinero para poder realizar impresiones braille en su hogar, comprando un
sistema operativo, una suite ofim\'atica, y un dispositivo con prestaciones
que exceden las necesidades del mismo.


% ----------------------------------------------------------------------------

\end{document}
