% Requerido por reglamento de trabajo final 
\chapter{Diagn\'ostico}

% Aca va una descripcion y analisis del problema abordado, un breve contenido
% de fundamento social y planteo de las necesidades.
% Incluir las razones qeu motivan este trabajo y se puede incluir una breve
% aproximacion historica.

% es recomendable dejarlo casi al final del trabajo

% Diagnóstico, en el mismo se mostrará la problemática actual, las
% dificultades a superar y todo aquello que se considere de interés y
% necesario para comprender el propósito que nos impulsa a tratar el tema.

En la actualidad existen muchas empresas que fabrican dispositivos de
impresi\'on braille, pero debido a el tama\~no reducido del nicho de mercado
en el que se encuentran, sus precios suelen ser muy elevados y no estan al
alcance del ciudadano medio.\ Y al pertenecer, de una forma u otra, a un sector
tecnol\'ogico, existe una gran competencia en cuanto al avance de sus
tecnologias. Esto conlleva a que fabriquen dispositivos con muchas
funcionalidades, prestaciones y de gran performace, dejando de lado dise\~nos
sencillos y meramente funcionales que harian al producto menos costoso.\\

Otro problema que presentan estos dispositivos, es que, a falta de estadares
de impresi\'on braille, cada fabricante provee su propia soluci\'on de
software que suele ser un costo extra en algunas ocaciones.\ 
Por este mismo motivo el soporte que proveen suele limitarse a un \'unico
sistema operativo\footnote{Normalmente Microsoft Windows.} forzando al usuario
a comprar una licencia del mismo e incluso en muchos casos un \'unico
procesador de texto\footnote{Normalmente Word de la suite Microsoft Office.}.\\

Se encuentra tambien dicha industria embebida en modelos de desarrollo
privativo, haciendo imposible al usuario final a agregar sus propios cambios
basandose en sus necesidades particulares.\ Si bien el modelo de desarrollo
privativo es uno de los mas usados en todas las industrias, existen varias que
se encuentran, ya sea en etapas de exploraci\'on o producci\'on, trabajando con
modelos \emph{open-source}\footnote{Del ingl\'es \emph{Codigo Abierto}} o
incluso \emph{Free Software}\footnote{Del ingl\'es \emph{Software Libre}},
siendo esto un lujo que la industria de dispositivos de impresion braille no
puede darse debido mayormente a su tama\~no.\\

% Todo esto ###### <---buscar una palabra mejor
Todo esto termina en que los usuarios finales deben gastar una importante suma
de dinero para poder realizar impresiones braille en su hogar, comprando un
sistema operativo, una suite ofim\'atica, y un dispositivo con prestaciones
que exceden las necesidades del mismo.
