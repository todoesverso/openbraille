\chapter{Diagn\'ostico}

En la actualidad existen muchas empresas\footnote{Ver \fullref{cap:market}} que
fabrican dispositivos de
impresi\'on braille, pero debido a el tama\~no reducido del nicho de mercado
en el que se encuentran, sus precios suelen ser muy elevados y no est\'an al
alcance del ciudadano medio.\ Y al pertenecer, de una forma u otra, a un sector
tecnol\'ogico, existe una gran competencia en cuanto al avance de sus
tecnolog\'ias. Esto conlleva a que fabriquen dispositivos con muchas
funcionalidades, prestaciones y de gran performance, dejando de lado dise\~nos
sencillos y meramente funcionales que har\'ian al producto menos costoso.\\

Otro problema que presentan estos dispositivos, es que, a falta de est\'andares
de impresi\'on braille, cada fabricante provee su propia soluci\'on de
software que suele ser un costo extra en algunas ocasiones.\ 
Por este mismo motivo el soporte que proveen suele limitarse a un \'unico
sistema operativo\footnote{Normalmente Microsoft Windows.} forzando al usuario
a comprar una licencia del mismo e incluso en muchos casos un \'unico
procesador de texto\footnote{Normalmente Word de la suite Microsoft Office.}.\\

Se encuentra tambi\'en dicha industria embebida en modelos de desarrollo
privativo, haciendo imposible al usuario final poder agregar cambios
bas\'andose en sus necesidades particulares.\ Si bien el modelo de desarrollo
privativo es uno de los m\'as usados en todas las industrias, existen varias
que
se encuentran, ya sea en etapas de exploraci\'on o producci\'on, trabajando con
modelos de \emph{C\'odigo Abierto}\footnote{Del ingl\'es \emph{open-source}} o
incluso \emph{Software Libre}\footnote{Del ingl\'es \emph{ Free Software}},
siendo esto un lujo que la industria de dispositivos de impresi\'on braille no
puede darse debido mayormente a su tama\~no.\\

Las problem\'aticas antes planteadas hacen que un posible mercado nacional de
estas tecnolog\'ias sea pr\'acticamente imposible, por lo que las impresoras
braille deben ser adquiridas en el exterior o mediante un importador.\\

% Todo esto ###### <---buscar una palabra mejor
Todo esto conlleva a que los usuarios finales deben gastar una importante suma
de dinero para poder realizar impresiones braille en su hogar, comprando un
sistema operativo, una suite ofim\'atica, y un dispositivo con prestaciones
que exceden las necesidades del mismo.
