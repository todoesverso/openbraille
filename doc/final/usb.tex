\chapter{USB}
El \emph{Universal Serial Bus}(USB\footnote{V\'ease - http://www.usb.org}) es
un bus serial estandar que hace de interfaz entre un dispositivo y una
computadora.
Dicha estandarizaci\'on es llevada acabo por el \emph{F\'orum de
Implementadores de USB} (USB Implementers Forum, USB-IF\footnote{V\'ease -
http://www.usb.org/about}). 
Actualmente el est\'andar se encuentra en su versi\'on 3.0\footnote{V\'ease -
http://www.usb.org/developers/docs/}, aunque la mayoria de las
implementaciones comerciales solo soportan el estandar \emph{2.0}.

\section{Historia}
%No me gusta como titulo 'historia' habria que pensar otra cosa

\section{Decripci\'on}
% Aca iria el funcionamiento del USB, difierenciado por \subsection{}
La especificaci\'on USB provee una serie de atributos con los cuales se pueden
implementar dispositivos segun el precio/rendimiento deseado.

\subsection{Caracter\'isticas}
Del amplio rango de caracter\'isticas definidas en el estandar, es posible
agruparlas en ocho categorias:
% Reescribir lo de arriba

\begin{itemize}
 \item Facilidad de uso para el usuario final
 \item Amplio rango de aplicaciones
 \item Ancho de banda is\'ocrono
 \item Flexibilidad
 \item Robustez
 \item Sinergia con la inustria de la PC
 \item Implementaci\'on de bajo costo
 \item De arquitectura actualizable
\end{itemize}

La combinaci\'on de estas caracteristicas hacen a la versatilidad del
estandar, ya que permiten implementar todo tipo de dispositivos segun la
carga, versatilidad, eficiencia, precio, velocidad y demas atributos que
puedan inferir en un dise\~no.

