\chapter{Conclusi\'on}

Ponerse como objetivo primario el reducir costos (sea cual fuere el motivo de
ello), es sin duda alguna una de las dificultades, que de una forma u otra
terminan manifestandose como t\'ecnicas, m\'as arduras de un trabajo de
ingenier\'a. Sin embargo las ganancias que se obtienen son invaluables.\\

Existen en la actualidad soluciones provistas por diversas empresas para casi
cualquier problema de la ingenier\'ia que requieren poco esfuerzo de
aprendizaje e incluso, en muchos casos, sus costos traen aparejado un retorno
de inversi\'on principalmente al reducir el tiempo de desarrollo.\
No obstante la principal desventaja de este tipo de soluciones es que abstraen
de manera significativa al ingeniero de las tecnolog\'ias y est\'andares
subyacentes, y suelen condicionarlo e incluso generar dependencia.\\

EL haber comenzado el trabajo sin ninguna inclinaci\'on hacia una metodologia
o herramienta en particular permiti\'o hacer un analisis objetivo del mismo
focalizandose en los objetivos finales y sin arrastrar condicionamientos
o limitaciones durante el desarrollo.\ 
De esta manera fue posible evaluar las opciones y tomar deciciones
confidentes.\\

La enorme cantidad de informacion recolectada para este trabajo fue tanto una
gran fuente de ayuda t\'ecnica como de aprendizaje en general, puesto que
todos los puntos tratados rozan tangencialemente diversas areas de gran
interes para el autor.\\

Durante el desarrollo del trabajo, cada uno de los objetivos propuestos fueron
siendo superados mediante un proceso de investigaci\'on, analisis e
implementaci\'on.\
Aunque en mucho casos la complejidad de ciertos topicos y la escasez de
informaci\'on hicieron del trabajo una tarea ardua, los mismos se re
interpretaron como desafios y fueron superados.\\




