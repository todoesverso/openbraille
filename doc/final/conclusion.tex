\chapter{Conclusi\'on} %%%%%%%%%%%%%%%%%%%%%%%%%%%%%%%%%%%%%%%%%%%%%%%%%%%%%%%%
%%%%%%%%%%%%%%%%%%%%%%%%%%%%%%%%%%%%%%%%%%%%%%%%%%%%%%%%%%%%%%%%%%%%%%%%%%%%%%%
%%%%%%%%%%%%%%%%%%%%%%%%%%%%%%%%%%%%%%%%%%%%%%%%%%%%%%%%%%%%%%%%%%%%%%%%%%%%%%%

Ponerse como objetivo primario el reducir costos (sea cual fuere el motivo de
ello), es sin duda alguna una de las dificultades, que de una forma u otra
terminan siendo t\'ecnicas, m\'as arduras de un trabajo de ingenier\'a. Sin
embargo las ganancias que se obtienen son invaluables.\\

Existen en la actualidad soluciones provistas por diversas empresas para casi
cualquier problema de la ingenier\'a que requieren poco esfuerzo de aprendizaje
e incluso, en muchos casos, sus costos traen aparejado un retorno de
inversi\'on principalmente al reducir el tiempo de desarrollo.\
No obstante la principal desventaja de este tipo de soluciones es que abstraen
de manera significativa al ingeniero de las tecnolog\'ias y est\'andares
subyacentes, y suelen condicionarlo e incluso generar dependencia.\\



