\chapter{Pr\'ologo} %%%%%%%%%%%%%%%%%%%%%%%%%%%%%%%%%%%%%%%%%%%%%%%%%%%%%%%%%%%
%%%%%%%%%%%%%%%%%%%%%%%%%%%%%%%%%%%%%%%%%%%%%%%%%%%%%%%%%%%%%%%%%%%%%%%%%%%%%%%
% Prólogo, donde se definan sus alcances, propósitos y objetivos.
% Requerido por reglamento de trabajo final 

Durante el transcurso del trabajo se describe el desarrollo e implementaci\'on
de la parte electr\'onica de una impresora braille. El objetivo final del
mismo es proveer una posible soluci\'on de bajo costo, tanto de producc\'ion
como de armado. Para ello, se estudiar\'an y analizar\'an todas aquellas
alternativas disponibles que puedan satisfacer los requerimientos
planteados.\\

La principal premisa con que se aborda el desarrollo de este trabajo es que
mediante soluciones b\'asicas y simples, es posible concretar dise\~nos
robustos y funcionales, siempre y cuando se hayan elegido las herramientas y
condiciones correctas. Para lograr esto, gran parte del trabajo se centra en
investigar alternativas de desarrollo no convencionales en la industria
electr\'onica.
