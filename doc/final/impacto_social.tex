\chapter{Impacto Social}

Todo el proyecto se encuentra bajo licencias libres, por lo que cualquier
persona o empresa interesada en \'este puede descargarlo de internet y
comenzar a usarlo. Esto implica que cualquier persona, empresa u organismo
p\'ublico o privado puede:

\begin{itemize}
 \item Usarlo con fines educativos
 \item Usarlo con fines personales
 \item Usarlo con fines lucrativos
\end{itemize}

Y justamente debido a la licencia elegida, la misma debe ser usada en cualquier
trabajo derivado, asegurando as\'i la continuidad de la libertad del mismo.\\

Como se mencion\'o anteriormente, todas las herramientas de software usadas
para el desarrollo son libres que, conjuntamente con la licencia del trabajo,
generan una sucesi\'on de eventos que favorecen econ\'omicamente a todos los
participes de \'este proceso.\
El desarrollador no necesita pagar por licencias de software en las
herramientas necesarias para trabajar con este proyecto, ni tampoco pagar para
obtener el c\'odigo y los dise\~nos aqu\'i presentados, esto logra que no se
transfieran ciertos costos al producto final mas que los del trabajo explicito
del desarrollador. Luego \'este puede vender su trabajo a una empresa que
productize el trabajo, quien habr\'a obtenido un producto de bajo costo y por
ende podr\'a venderlo a precios razonables de igual manera. Finalmente el
usuario final obtiene un producto barato que no requiere comprar software
alguno para funcionar, economizando all\'i tambi\'en.\\

En una primera instancia, y debido al mercado aun en un estado de cierta
incertidumbre y con competencia pr\'acticamente nula\footnote{Refiri\'endonos
al
mercado nacional}, productos basados en este proyecto sufrir\'ian un impacto en
el precio final, no obstante serian precios perfectamente afrontables por
entidades p\'ublicas o privadas como escuelas, universidades y bibliotecas.\ 
Pero en el escenario hipot\'etico en el que el mercado nacional se crezca, los
precios tender\'ian a bajar al punto de poder ser afrontados por personas de
recursos limitados.\\ 

De una forma u otra, el solo hecho de incrementar la cantidad de impresoras
braille en Argentina es una necesidad innegable, ya sea a nivel institucional u
hogare\~no.\\