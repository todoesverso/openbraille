% Poner primero, antes del codigo, un diagrama de flujo del mismo

Para comenzar una comunicaci\'on USB con la API \emph{libusb} es preciso
primero descubrir el dipositivo, esto se logra con:

\begin{lstlisting}
struct usb_bus *busses;
    
usb_init();
usb_find_busses();
usb_find_devices();
    
busses = usb_get_busses();
\end{lstlisting}

Luego de esto ya se poseen los \emph{busses} USB del sitema. Es preciso luego
iterar sobre cada uno de ellos para encontrar el dispositivo especifico de la
siguiente manera:

\begin{lstlisting}
struct usb_bus *bus;
int c, i, a;
    
/* ... */
    
for (bus = busses; bus; bus = bus->next) {
  struct usb_device *dev;
    
    for (dev = bus->devices; dev; dev = dev->next) {
        /* Buscar el dispositivo por VendorID */
        if (dev->descriptor.idVendor == MY_ID) {
            /* Buscar el dispositivo por ProductID */
            if (dev->descriptor.idProduct==USBPRINTER){
            /* Abrir dispositivo */
            udev = usb_open(dev);
            /* Reclamar la interfaz */
            ret = usb_claim_interface(udev,0); 

            /* Programa */
            }	...
        }
    }
}
\end{lstlisting}

Una limitaci\'on intr\'inseca de la libreria es que se requiere abrir tantas
instancias del dispositivo como interfaces se desee usar de \'el.\\
