% Poner primero, antes del c\'odigo, un diagrama de flujo del mismo

Para comenzar una comunicaci\'on USB con la API \emph{libusb} es preciso
primero descubrir el dispositivo, esto se logra con:

\begin{lstlisting}
struct usb_bus *busses;
    
usb_init();
usb_find_busses();
usb_find_devices();
    
busses = usb_get_busses();
\end{lstlisting}

Luego de esto ya se poseen los \emph{busses} USB del sistema. Es preciso luego
iterar sobre cada uno de ellos para encontrar el dispositivo especifico de la
siguiente manera:

\begin{lstlisting}
struct usb_bus *bus;
int c, i, a;
    
/* ... */
    
for (bus = busses; bus; bus = bus->next) {
  struct usb_device *dev;
    
    for (dev = bus->devices; dev; dev = dev->next) {
        /* Buscar el dispositivo por VendorID */
        if (dev->descriptor.idVendor == MY_ID) {
            /* Buscar el dispositivo por ProductID */
            if (dev->descriptor.idProduct==USBPRINTER){
            /* Abrir dispositivo */
            udev = usb_open(dev);
            /* Reclamar la interfaz */
            ret = usb_claim_interface(udev,0); 

            /* Programa */
            }	...
        }
    }
}
\end{lstlisting}

Una limitaci\'on intr\'inseca de la librer\'ia es que se requiere abrir tantas
instancias del dispositivo como interfaces se desee usar de \'el.\\

Luego de esto, ya se ha inicializado la comunicaci\'on con el dispositivo USB,
resta solo hacer uso de las funciones que la API provee para comunicarse con
el mismo; estas son:

\begin{lstlisting}
int usb_bulk_write(usb_dev_handle *dev, int ep, const char *bytes, int size,
int timeout);
int usb_bulk_read(usb_dev_handle *dev, int ep, char *bytes, int size, int
timeout);
\end{lstlisting}

Estas dos funciones son la \'unicas necesarias para llevar a cabo las tareas
especificas que debe cumplir el driver.\\

Tambi\'en como parte del trabajo se desarroll\'o un modulo para manejo de
errores que hace uso de un peque\~no sintetizador de voz\footnote{V\'ease -
\url{http://espeak.sourceforge.net/}} para decir los mensajes por los parlantes
de la computadora. Este modulo es capaz de manejar varios idiomas. El c\'odigo
del mismo se encuentra en el ap\'endice en \fullref{cap:errors}.