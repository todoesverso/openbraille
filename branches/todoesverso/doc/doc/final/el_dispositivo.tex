% El objetivo de este capitulo es presentar a grandes rasgos el dispositivo 
% y luego sus detalles. Sus caracteristicas  mas importantes, explicar el 
% funcionamiento, alternativas, diagramas, comparaciones, ventajas y 
% desventajas de cada opcion, como presentar tambien los detalles pertinentes,
% los factores principales y definir prioridades de dise\~no

\chapter{El dispositivo}
% Comentario de este capitulo 
\section{Introduccion} 
% Breve introduccion para presentar el dispositivo visto como prototipo que
% tiene que cumplir determinada funcion, planteando los requerimientos y
% ampliando en detalles gradualmente a medida que se avanza en la lectura
% el enfoque tiende a ser teorico antes de abordar cualqueir conclusion parcial
% practica
La impresora Braille o embosser (estampadora) es un dispositivo que realiza un
estampado en relieve de arreglo de puntos sobre un papel ductil, de manera 
autom\'atica (es decir no-manual).
% incluir un diagrama en bloques simplificado del sistema completo aqui mismo.
\section{El sistema Braille}
% Draft inicial de esta seccion. Ampliarla de acuerdo a la relevancia e incluir
% figuras y graficos
El sistema Braille es un m\'etodo utilizado por personas ciegas para leer y 
escribir. Fue ideado en 1821 por el franc\'es Louis Braille
Se basa en un m\'etodo de comunicaci\'on desarrollado y perfeccionado por 
Charles Barbier, en respuesta a la demanda de Napole\'on, de un c\'odigo que
los soldados pudieran usar para comunicarse en silencio y sin luz en la noche.
Se lo llam\'o Night writing. El sistema de Barbier era demasiado complejo para
los soldados de aprender, y fue rechazada por los militares. En 1821 visit\'o
el Instituto Nacional para Ciegos, en Par\'is, donde conoci\'o a Louis Braille,
qui\'en identific\'o el mayor defecto del c\'odigo: el dedo de la mano humana
no puede abarcar todo el s\'imbolo sin moverse, y as\'i no puede pasar
r\'apidamente de un s\'imbolo a otro.
Su modificaci\'on fue utilizar una celda de 6 puntos (el sistema Braille) que 
revolucion\'o la comunicaci\'on escrita de los ciegos.
Cada c\'elula (o celda) braille o car\'acter se compone de seis posiciones de 
puntos, dispuestos en un rect\'angulo que contiene dos columnas de tres puntos
cada uno. Un punto puede ser colocado en alguna de las seis posiciones para
formar sesenta y cuatro ($2^{6}$) permutaciones, incluido el arreglo de puntos
que no se coloca. Una permutaci\'on puede ser descrita nombrando las posiciones
en que se disponen los puntos: Las posiciones est\'an universalmente numerados
de 1 a 3, de arriba a abajo, a la izquierda, y 4 a 6, de arriba a abajo, a la
derecha. 

% Por ejemplo, puntos 1-3-4 describen una celda con tres puntos planteados, en 
% la parte superior e inferior de la columna de la izquierda y en la parte
% superior de la columna de la derecha, es decir, la letra m. 

Las l\'ineas horizontales de texto en Braille est\'an separados por un espacio
a fin de que los puntos de una l\'inea puede ser diferenciada de la de texto
en braille por encima y por debajo. La puntuacion est\'a representada por su
propio conjunto de caracteres \'unico.
% incluir figuras para facilitar la explicacion

\section{Impactadoras y el proceso de estampado} 
El estampado es el proceso de crear una imagen tridimensional o dise~no en
papel y otros materiales d\'octiles. Normalmente se realiza con una presi\'on
en el papel. 
Esto se logra mediante el uso de un molde de metal (hembra) y un contra
molde(macho) que encajan entre s\'i. Esta presi\'in moldea mientras aumenta el
nivel de la imagen superior a la del sustrato, acutando como impactadoras.
En general, la mayor\'ia de los tipos de papel y tableros pueden ser en relieve
y no hay restricciones en el tama~no.

%\section {Nueva_seccion}
% nuevamente modificar el titulo anterior si no es apropiado, puede encontrarse
% una referencia bibliografica para respaldar la eleccion (conveniente pero no
% excluyente)
%
%
% 
