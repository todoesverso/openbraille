% Requerido por reglamento de trabajo final 
\chapter{Resumen}

% Resumen, donde se describe brevemente el trabajo realizado-
% Una idea inicial de resumen es la que se escribe a continuacion
% se pueden repetir lo incluido en diagnostico y objetivos siempre que no tenga
% una extension considerable que sobrepase la caracteristica de resumen

El siguiente trabajo describe los pasos para construir un dispositivo simple
para la impresion de texto Braille, mediante la utilizacion de SL durante todas
las etapas del proyecto. El dispositivo tiene las funcionalidades esenciales de
una impresora Braille, suficientes para lograr el estampado de los puntos del
codigo sobre el papel Braille. El trabajo se focaliza sobre la aplicacion de
herramientas libres en la obtencion de un elemento que hereda estas
caracteristicas, contemplado en el tipo de licencia utilizado % para este
%trabajo. 
Se hara refrencia a la impresora Braille como embosser en el transcurso de este
trabajo.
% embosser se traduciria a estampadora o gofradora, dispositivo de gofrado

%

% Pueden agregarse secciones si es necesario
%\section{Ejemplo} 
